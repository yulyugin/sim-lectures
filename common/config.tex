\documentclass{beamer}

% Base packages
\usepackage{fontspec}
\defaultfontfeatures{Scale=MatchLowercase, Mapping=tex-text}

\usepackage{amsfonts}
\usepackage{amsmath}
\usepackage{longtable}
\usepackage{csquotes}
\usepackage{standalone}

\usepackage{graphicx}
\graphicspath{{../pictures/}}

\usepackage{tikz}
\usetikzlibrary{arrows,decorations.pathmorphing,backgrounds,positioning,fit,petri}
\newcommand{\inputpicture}[1]{\input{../drawings/#1}}

\usepackage{listings}
\lstset{language=C, basicstyle=\ttfamily, breaklines=true, keepspaces=true, keywordstyle=\color{blue}}

% Setup fonts
\newfontfamily\russianfont{CMU Serif}
\setromanfont{CMU Serif}
\setsansfont{CMU Sans Serif}
\setmonofont{CMU Typewriter Text}

% Be able to insert hyperlinks
\usepackage{hyperref}
\hypersetup{colorlinks=true, linkcolor=black, filecolor=black, citecolor=black, urlcolor=blue , pdfauthor=Grigory Rechistov <grigory.rechistov@phystech.edu> and Evgeny Yulyugin <yulyugin@gmail.com>, pdftitle=Программное моделирование вычислительнызх систем}
% \usepackage{url}

% Misc optional packages
\usepackage{underscore}
\usepackage{amsthm}

% A new command to mark not done places
\newcommand{\todo}[1][]{{\color{red}TODO\ #1}}

\newcommand{\abbr}{\textit{англ.}\ }

\subtitle{Курс «Программное моделирование вычислительных систем»}
\subject{Лекция}
\date{\today}
\pgfdeclareimage[height=0.5cm]{ilab-logo}{../common/ilab-noletters.png}
\logo{\pgfuseimage{ilab-logo}}

\typeout{Copyright 2014 Grigory Rechistov and Evgeny Yulyugin}

\usetheme{Berlin}
\setbeamertemplate{navigation symbols}{}%remove navigation symbols
