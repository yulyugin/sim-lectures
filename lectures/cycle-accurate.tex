\documentclass{beamer}

\usepackage{amsfonts}
\usepackage{amsmath}
\usepackage{longtable}
\usepackage{csquotes}
\usepackage{standalone}

\usepackage{graphicx}
\graphicspath{{../pictures/}}

\usepackage{tikz}
\usetikzlibrary{shapes, calc, arrows, decorations.markings,
  decorations.pathmorphing, decorations, patterns, chains, snakes,
  backgrounds, positioning, fit, petri}
\newcommand{\inputpicture}[1]{\input{../drawings/#1}}

\usepackage{listings}
\lstset{language=C, basicstyle=\ttfamily, breaklines=true, keepspaces=true,
  keywordstyle=\color{blue}}

\usepackage{bytefield}

\usefonttheme{professionalfonts}
\usefonttheme{serif}
\usepackage{fontspec}
\setromanfont{CMU Serif}
\setsansfont{CMU Sans Serif}
\setmonofont{CMU Typewriter Text}

\usepackage{hyperref}
\hypersetup{colorlinks=true, linkcolor=black, filecolor=black, citecolor=black,
  urlcolor=blue , pdfauthor=Evgenii Iuliugin <yulyugin@gmail.com>,
  pdftitle=Fundamentals of Full-Platform Simulation}

\usepackage{underscore}
\usepackage{amsthm}

\subtitle{Fundamentals of Full-Platform Simulation}
\subject{Lecture}
\date{\today}

\author[Evgenii Iuliugin]{
  Evgenii Iuliugin, \small{\href{mailto:yulyugin@gmail.com}{yulyugin@gmail.com}}}
\typeout{Copyright 2021 Evgenii Iuliugin}

\usetheme{Madrid}
\setbeamertemplate{navigation symbols}{}

\makeatletter
\setbeamertemplate{footline}{%
  \leavevmode%
  \hbox{%
    \begin{beamercolorbox}[wd=.15\paperwidth,ht=2.25ex,dp=1ex,center]{author in head/foot}%
      \usebeamerfont{author in head/foot}\insertshortauthor\expandafter\ifblank\expandafter{\beamer@shortinstitute}{}{~~(\insertshortinstitute)}
    \end{beamercolorbox}%
    \begin{beamercolorbox}[wd=.77\paperwidth,ht=2.25ex,dp=1ex,center]{title in head/foot}%
      \usebeamerfont{title in head/foot}\insertshorttitle
    \end{beamercolorbox}%
  }%
  \begin{beamercolorbox}[wd=.08\paperwidth,ht=2.25ex,dp=1ex,right]{date in head/foot}%
    \usebeamerfont{date in head/foot}%
    \usebeamertemplate{page number in head/foot}%
    \hspace*{2ex}
  \end{beamercolorbox}
  \vskip0pt%
}
\makeatother

\newcommand{\startslides}{
  {\setbeamertemplate{footline}{}
  \begin{frame}
      \maketitle
  \end{frame}
  }

  \addtocounter{framenumber}{-1}

  \begin{frame}{\inserttitle}
      \tableofcontents
  \end{frame}
}

\newcommand{\finalslide}{{
  \setbeamertemplate{footline}{}

  \begin{frame}

  {\huge{Thank you!}\par}

  \vfill
  Slides and material are available at
  \url{https://github.com/yulyugin/sim-lectures}
  \vfill

  \tiny{\textit{Note}: All trademarks are the property of their respective
    owners. The presented point of view reflects the personal opinion of
    the author.

    %All the materials are licensed under the Creative Commons
    %Attribution-NonCommercial-ShareAlike 4.0 Worldwide. To view a copy of this
    %license, visit \url{http://creativecommons.org/licenses/by-nc-sa/4.0/}.
  }
  \end{frame}
}\addtocounter{framenumber}{-1}}

\title{Потактовая симуляция}

\begin{document}

\startslides

\begin{frame}{На прошлой лекции}
\begin{itemize}
    \item Модель, управляемая исполнением (функциональная модель)
    \item Модель, управляемая событиями (DES)\pause
    \item \textbf{Модель, моделирующая каждый такт} (time-stepped)
\end{itemize}
\end{frame}

\begin{frame}{Questions}
\begin{itemize}
\item Сколько бит в машинном слове? \pause
\item Что лучше — MMIO или PIO?\pause
\item Может ли архитектура быть и не little, и не big-endian?
\end{itemize}
\end{frame}

\section{Потактовая модель}

\begin{frame}{Что моделируем}
\centering

\includegraphics[width=\textwidth]{./mips-arch}

\tiny{\url{http://commons.wikimedia.org/wiki/File:MIPS_Architecture_(Pipelined).svg}}
\end{frame}

\begin{frame}{Проблемы}
\begin{itemize}
    \item Функциональная модель — не работает (слишком грубая)
    \item DES — применима, но неудобная абстракция
\end{itemize}

\inputpicture{des-long}

\end{frame}

\begin{frame}{Особенности}

\inputpicture{features}
\end{frame}

\begin{frame}{Проблемы}
\begin{itemize}
\item Длительность одной операции у разных узлов могут быть различными
\item Как проверять готовность «медленных» узлов?
\item Результаты обработки данных должны появляться не ранее, чем на такте, следующим за текущим
% \item Нельзя в произвольном порядке обновлять состояние блоков
\end{itemize}

\end{frame}

\section{Функции и порты}

\begin{frame}{Решение}
Отделим:
    \begin{itemize}
    \item Функции узлов
    \item Время, затрачиваемое на их выполнение
    \item Внутреннее состояние узлов
    \end{itemize}
\end{frame}

\begin{frame}{Функциональный элемент}
Результат готов «мгновенно» при наличии входных данных

\vfill
\centering
\inputpicture{pure-function}

\end{frame}

\begin{frame}{Порт}
Очередь фиксированной задержки

Ширина $N$ бит, задержка 1 такт

\vfill
\centering
\inputpicture{delay-line}

\end{frame}

\begin{frame}{Правило соединения}

\begin{itemize}
\item Функции не могут соединяться непосредственно друг с другом
\item Чередующиеся фазы симуляции:
\begin{enumerate}
    \item симуляция функций;
    \item симуляция передачи результатов
\end{enumerate}
\end{itemize}
\end{frame}

\begin{frame}{Модель с портами: фаза 1}

\centering
\inputpicture{cycle-phase1}

\end{frame}

\begin{frame}{Модель с портами: фаза 2}

\centering
\inputpicture{cycle-phase2}

\end{frame}


\section{Детали реализации}

\begin{frame}{Готовность данных}

\centering
\inputpicture{valid}

\end{frame}

\begin{frame}{Могут ли функциональные элементы иметь память?}

\centering
\inputpicture{state-storing}

\end{frame}

\begin{frame}{Композиция узлов}

\centering
\inputpicture{ports-compose}

\end{frame}

\begin{frame}{Связь функциональной и потактовой моделей}

\centering\includegraphics[width=0.8\textwidth]{functional-cycle-precise-connection}

\end{frame}

% \begin{frame}{Диаграмма моделируемой системы}
% 
% \end{frame}


\begin{frame}[allowframebreaks]{Литература}
\begin{thebibliography}{99}
    \bibitem{patterson-hennessy} Дэвид Паттерсон и Джон Хэннесси. Архитектура компьютера и проектирование компьютерных
систем. 4-е изд. Питер, 2012.

\bibitem{Asim} Joel Emer, Pritpal Ahuja, Eric Borch, Artur Klauser, Chi-Keung Luk, Srilatha Manne, Shubhendu S. Mukherjee,
Harish Patil, Steven Wallace, Nathan Binkert, Roger Espasa, Toni Juan. Asim: A Performance Model Framework // Computer 35 (2002), p. 68–76.

\bibitem{baida} Ю.В. Байда. Методы разработки и тестирования аппаратных потактовых моделей микропроцессоров на программируемых логических интегральных схемах. Дисс. к.т.н. — 2013
\end{thebibliography}
\end{frame}

\begin{frame}{На следующей лекции}
\centering

Параллельная симуляция, управляемая исполнением (MPonMP)

\end{frame}

\finalslide

\end{document}
