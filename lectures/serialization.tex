\documentclass{beamer}

\usepackage{amsfonts}
\usepackage{amsmath}
\usepackage{longtable}
\usepackage{csquotes}
\usepackage{standalone}

\usepackage{graphicx}
\graphicspath{{../pictures/}}

\usepackage{tikz}
\usetikzlibrary{shapes, calc, arrows, decorations.markings,
  decorations.pathmorphing, decorations, patterns, chains, snakes,
  backgrounds, positioning, fit, petri}
\newcommand{\inputpicture}[1]{\input{../drawings/#1}}

\usepackage{listings}
\lstset{language=C, basicstyle=\ttfamily, breaklines=true, keepspaces=true,
  keywordstyle=\color{blue}}

\usepackage{bytefield}

\usefonttheme{professionalfonts}
\usefonttheme{serif}
\usepackage{fontspec}
\setromanfont{CMU Serif}
\setsansfont{CMU Sans Serif}
\setmonofont{CMU Typewriter Text}

\usepackage{hyperref}
\hypersetup{colorlinks=true, linkcolor=black, filecolor=black, citecolor=black,
  urlcolor=blue , pdfauthor=Evgenii Iuliugin <yulyugin@gmail.com>,
  pdftitle=Fundamentals of Full-Platform Simulation}

\usepackage{underscore}
\usepackage{amsthm}

\subtitle{Fundamentals of Full-Platform Simulation}
\subject{Lecture}
\date{\today}

\author[Evgenii Iuliugin]{
  Evgenii Iuliugin, \small{\href{mailto:yulyugin@gmail.com}{yulyugin@gmail.com}}}
\typeout{Copyright 2021 Evgenii Iuliugin}

\usetheme{Madrid}
\setbeamertemplate{navigation symbols}{}

\makeatletter
\setbeamertemplate{footline}{%
  \leavevmode%
  \hbox{%
    \begin{beamercolorbox}[wd=.15\paperwidth,ht=2.25ex,dp=1ex,center]{author in head/foot}%
      \usebeamerfont{author in head/foot}\insertshortauthor\expandafter\ifblank\expandafter{\beamer@shortinstitute}{}{~~(\insertshortinstitute)}
    \end{beamercolorbox}%
    \begin{beamercolorbox}[wd=.77\paperwidth,ht=2.25ex,dp=1ex,center]{title in head/foot}%
      \usebeamerfont{title in head/foot}\insertshorttitle
    \end{beamercolorbox}%
  }%
  \begin{beamercolorbox}[wd=.08\paperwidth,ht=2.25ex,dp=1ex,right]{date in head/foot}%
    \usebeamerfont{date in head/foot}%
    \usebeamertemplate{page number in head/foot}%
    \hspace*{2ex}
  \end{beamercolorbox}
  \vskip0pt%
}
\makeatother

\newcommand{\startslides}{
  {\setbeamertemplate{footline}{}
  \begin{frame}
      \maketitle
  \end{frame}
  }

  \addtocounter{framenumber}{-1}

  \begin{frame}{\inserttitle}
      \tableofcontents
  \end{frame}
}

\newcommand{\finalslide}{{
  \setbeamertemplate{footline}{}

  \begin{frame}

  {\huge{Thank you!}\par}

  \vfill
  Slides and material are available at
  \url{https://github.com/yulyugin/sim-lectures}
  \vfill

  \tiny{\textit{Note}: All trademarks are the property of their respective
    owners. The presented point of view reflects the personal opinion of
    the author.

    %All the materials are licensed under the Creative Commons
    %Attribution-NonCommercial-ShareAlike 4.0 Worldwide. To view a copy of this
    %license, visit \url{http://creativecommons.org/licenses/by-nc-sa/4.0/}.
  }
  \end{frame}
}\addtocounter{framenumber}{-1}}


\author{Grigory Rechistov \thanks{grigory.rechistov@intel.com}}
\title{Serialization of Simulated State}
\institute{Intel Corporation}

\begin{document}

\startslides

\begin{frame}{About the Presenter}
\begin{itemize}
\item At MIPT: 2004−2016
\item At Intel: 2007−2021
\item Read this course in 2010-2016
\end{itemize}
\end{frame}


% \begin{frame}{Questions}
% \begin{itemize}
% \item 
% \end{itemize}
% \end{frame}

\section{Definitions}

\begin{frame}{Serialization in sofware}

From \href{https://en.wikipedia.org/wiki/Serialization}{Wikipedia}: \par process of translating a data structure or object state into a format that can be stored (for example, in a file or memory data buffer) or transmitted (for example, over a computer network) and reconstructed later (possibly in a different computer environment).

\end{frame}

\begin{frame}{Serialization in software simulation}

\begin{itemize}
    \item Usually called checkpoints, savepoints, snapshots etc.
    \item Conversion of target architectural state (embodied in host memory image) to a serialized format (embodied as host files), and in reverse.
    \item May or may not include other state, e.g. simulator's own state.
\end{itemize}

\end{frame}

\section{Usage cases}

\begin{frame}{Usage cases}
\begin{itemize}
    \item Starting over at a certain point of interest
    \item Not wasting time on doing the same pre-work
    \item Moving from one host to another
    \item Sharing with other humans
    \item Parallelizing dependent simulations
    \item Cool tricks such as turning back time
\end{itemize}

\pause

\includegraphics[width=0.5\textwidth]{save-point-device.png}

\end{frame}

\section{What to save; what to drop}

\begin{frame}{Target state to save}

Things describing the architectural state of target system usually have to be saved.

\begin{itemize}
    \item Register state of devices
    \item Indexed storage (memory banks, disks, flash memory etc.) \pause
    \item Connections between target devices via references
\end{itemize}

\end{frame}

\begin{frame}{Simulator/host state to (not) save}

Usually not much outside of target state should be saved.

\begin{itemize}
    \item Simulator's internal computing environment, e.g. variables. Usually it leaks host-specific data
    \item Debugger environement, e.g. breakpoints. Usually it is under-specified
    \item (DO NOT DO IT) Simulator's own code (binaries). Totally un-portable
\end{itemize}

\end{frame}

\section{Implementation}

\begin{frame}{Implementations}
\begin{itemize}
\item Many langugages support object serialization, so you get it for free.
\item Python, Java, .NET, Go\pause
\item No such luck with C/C++
\item Simply \texttt{memcpy(obj, buffer, sizeof(obj_t))} would not work (see below why)
\item A lot of repetitive manual work, unless you use ready libraries (such as Boost.Serialization)
\item Usually resolved by registering checkpointable state with a centralized "authority"
\end{itemize}

\end{frame}

\section{Demo}

\begin{frame}{Demo}

Checkpoints in Simics

\begin{enumerate}
    \item Boot QSP
    \item Create checkpoint
    \item Close Simics
    \item Load checkpoint
    \item Load checkpoint again (different namespace)
    \item Inspect the checkpoint (config.gz file, CRAFF files)
\end{enumerate}

\end{frame}



\section{Tricky aspects}

\begin{frame}{Saving references}

\begin{itemize}
\item Connections (references) between objects are represented by pointers
\item Pointers are fragile: host-specific, change between runs. They cannot be reliably saved
\end{itemize}

\begin{itemize}
    \item At save time, replace pointers with persistent object handles
    \item At restore time, first create objects preserving their handles
    \item Reconnect them with each other using the handles
\end{itemize}

\end{frame}

\begin{frame}{Pitfalls}
\begin{itemize}
    \item Not saving enough state
    \item Saving more state than necessary
    \item Saving in a host-dependent format
    \item Restoring using non-compatible simulator
\end{itemize}
\end{frame}

\begin{frame}{How to tell if you preserved enough?}

A simulation restored from a checkpoint should continue identically, as if it has never been interrupted.

E.g. saving only persistent storage (disks) is clearly not enough. In memory and in-register state of all programs should be preserved as well.

How much is enough? Usually verified with tests.

\begin{itemize}
    \item Differential runs: run a restored system alongside "fresh" one, continue them both and periodically compare their states.
    \item Restore state, then immediately save again, and compare contents of checkpoints
\end{itemize}

\end{frame}

\begin{frame}{Saving too much?}

Certain state should not be saved. Risks: increase in checkpoint size, save/restore time, or even incorrect operation.

Symptoms:

\begin{itemize}
    \item Serialziation of state not relevant to target architecture
    \item Saving of cached values that can be derived from other serialized state
    \item Saving of read-only values
    \item Usage of special "pseudo" items
    \item Order of serialization matters (violation of \href{https://en.wikipedia.org/wiki/Law_of_Demeter}{Law of Demeter}?)
\end{itemize}

Do not save state "just in case"

\pause "There are two hard problems in computer science: 1) naming entities, 2) maintaining cache, and 3) off-by-one mistakes".

\end{frame}

\begin{frame}[fragile]{Host-dependent format}

\begin{lstlisting}
uint8_t 
get_least_significant_byte_of_intptr(int32_t *ptr) 
{
    return *(uint8_t*)ptr;
}
\end{lstlisting}

\pause
Issues: byte order, integer representation, octet vs byte.

\pause
\begin{itemize}
    \item Representation of numbers (integer and floating-point)
    \item Byte order
    \item Object and structures layout
    \item Pointer format and fragility
\end{itemize}

Primitives must have host-independent representation. 
Complex objects should be recursively split into serializable primitives.

\end{frame}

\begin{frame}{Restoring in different build of simulator}

Unless explicitly taken care of, it won't work:

\begin{itemize}
    \item Serialized primitives may have been renamed or changed layout
    \item Old items may be deleted
    \item New items may be added
\end{itemize}

The problem can be partially alleviated by manually-written converter code, or
using a versioning system for serialization format.
\end{frame}

\begin{frame}{Format of serialized data}

Many alternatives for structured data representation
\begin{itemize}
    \item Structured languages: XML, JSON, Yaml, …
    \item Binary formats: protobuf
    \item Custom formats
\end{itemize}

Human readability for structured data is nice to have. Excessive verbosity can be solved by using compressed container.

For unstructured data (indexed storage), human readability is usually not required, but compactness is desirable (sparse files).

\end{frame}


\begin{frame}{Breaking circular dependencies at restoration}

An object has references to other objects, so it can be restored only after they have been instantiated.
But what to do if there are circular references (A points to B, and B points back to A)?

Usually solved by restoration in two phases.

\begin{enumerate}
    \item All objects are created, and non-reference members are restored
    \item References to objects are restored. It is expected that objects are "almost" complete, e.g. has their interfaces available for querying.
\end{enumerate}

It is a bad thing if order of attribute or object restoration plays role. It means that objects are too tightly coupled.

\end{frame}

\begin{frame}{External interactions}

Suppose you want to preserve not an isolated quiescent system but a scenario that demonstrates a specific interaction with outside world.

\begin{enumerate}
    \item OS has booted
    \item User presses keyboard buttons to open a browser and enter URL
    \item Network packet exchange ensues, which brings an image of a website to the simulated screen.
\end{enumerate}

Both user and network are outside of your control.

In this case, a checkpoint should contain original system's state and a recording of "future" external input events alongside their timestamps.

Restoring such checkpoint involves replaying the inputs.

\end{frame}


\begin{frame}{Summary}
\begin{itemize}
\item Checkpoints are cool
\item Certain languages provide native serialization. Other languages have libraries for that. Otherwise, write your own and suffer
\item Pitfalls: pointers, host-dependent formats, object dependencies
\item Working on checkpoints helps understand what essentially constitutes the target system
\end{itemize}
\end{frame}

% \begin{frame}[allowframebreaks]{Bibliography}
% \begin{thebibliography}{99}
  % \bibitem{} \textit{todo authors}. TODO title

% \end{thebibliography}
% \end{frame}

\begin{frame}{On the Next Lecture}
Counters and how to fake them
\end{frame}

\finalslide

\end{document}
