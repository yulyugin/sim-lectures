% This file allows to produce either a separate PDF/PNG image
% See standalone documentation to understand underlying magic

\documentclass[tikz,convert={density=150,size=600,outext=.png}]{standalone}
\usetikzlibrary{shapes, calc, arrows, fit, positioning, decorations, patterns, decorations.pathreplacing, chains, snakes}
\input{../setup-web-fonts}
\input{../setup-packages}
\graphicspath{{../pictures/}} % path to pictures, trailing slash is mandatory.

% The actual drawing follows
\begin{document}
\begin{tikzpicture}[scale=1.0, >=latex, font=\small]

\begin{scope}[text width=1.4cm]
    \node[draw, rounded corners] (file1) {file1.dml};
    \node[draw, rounded corners, below= 0.75cm of file1.south west, anchor=north west] (file-etc) {...};
    \node[draw, rounded corners, below= 0.75cm of file-etc.south west, anchor=north west] (fileN) {fileN.dml};
\end{scope}
    
\node[draw, right = 0.5cm of fileN-h, text width=1.8cm, align=center] (compiler) {DMLC \\ компилятор};

\begin{scope}[text width=1.cm]
    \node[draw, rounded corners, below right= 0cm and 3.2cm of file1] (file1-c) {file1.c};
    \node[draw, rounded corners, above right= 0cm and 3.2cm of fileN] (fileN-c) {fileN.c};
\end{scope}

\node[draw, right = 2.6cm of compiler,] (gcc) {GCC};

\node[draw, rounded corners, right = 0.5cm of linker, align=center, fill=black!10] (exe) {Исполняемый\\файл};

\draw[->] (file1.east) -- (compiler);
\draw[->] (file-etc.east) -- (compiler);
\draw[->] (fileN.east) -- (compiler);

\draw[->] (compiler) -- (file1-c);
\draw[->] (compiler) -- (fileN-c);

\draw[->] (file1-c) -- (gcc);
\draw[->] (fileN-c) -- (gcc);

\draw[->] (gcc) -- (exe);

\end{tikzpicture}

\end{document}
